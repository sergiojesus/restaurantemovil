\chapter{Marco te\'orico}



\section{Herramientas utilizadas}

\subsection{Java}
Java es un lenguaje de programación y una plataforma informática comercializada por primera vez en 1995 por Sun Microsystems. Hay muchas aplicaciones y sitios web que no funcionarán a menos que tenga Java instalado y cada día se crean más. Java es rápido, seguro y fiable. 
\cite[Oracle Corporation. (2018)]{referencia7}


\subsection{android}
Android ofrece un completo framework de aplicaciones que te permite crear apps y juegos innovadores para dispositivos móviles en un entorno de lenguaje Java. Los documentos que se indican en la barra de navegación izquierda proporcionan detalles acerca de cómo crear apps usando diferentes API de Android.
\cite[Android. (2018)]{referencia6}



\subsection{Sql lite}
SQLite es un motor de base de datos SQL incorporado. A diferencia de la mayoría de las otras bases de datos SQL, SQLite no tiene un proceso de servidor por separado. SQLite lee y escribe directamente en archivos de disco ordinarios. Una base de datos SQL completa con múltiples tablas, índices, disparadores y vistas, está contenida en un solo archivo de disco.
	\cite[sqlLite. (2017)]{referencia1}

\subsection{material design}
Material Design es un sistema unificado que combina teoría, recursos y herramientas para crear experiencias digitales.
\cite[Material Design. (2017)]{referencia2}



\subsection{GIT}
GitHub es una plataforma de desarrollo colaborativo de software para alojar proyectos utilizando el sistema de control de versiones Git.
\cite[git hub,(2018)]{referencia3}


\subsection{UML}
El lenguaje UML es un estandar de OMG diseñado para visualizar, especificar, construir y documentar software orientado a objetos.
\cite[Grady Booch. (2015) )]{referencia5}


\subsection{Mock up}
Los mockups son muy útiles para que, cuando empieces el diseño final, no pierdas de vista el objetivo. Los bocetos sirven como brújulas durante el proceso de diseño. En Canva, tienes muchas herramientas de diseño para crear mockups de sitios web, de aplicaciones o de lo que quieras. No importa si estás trabajando como freelancer o si necesitas generar una gran cantidad de piezas, en Canva es muy rápido y fácil crear excelentes mockups para que tengas más tiempo para el diseño final.
\cite[canva. (2018)]{referencia4}