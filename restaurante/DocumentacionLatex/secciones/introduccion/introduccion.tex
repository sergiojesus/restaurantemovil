\chapter{Introducci\'on}
\section{Planteamiento del problema}

En un restaurante surge la necesidad de hacer reservaciones de una manera m\'as tecnol\'ogica y r\'apida ya que \'ultimamente ha tenido un problema con la administración de mesas reservadas.

Otra necesidad del restaurante es que no ha tenido el impacto que desear\'ia tener por lo que necesita dar a conocer sus platillos y promociones para as\'i atraer a m\'as personas al restaurante.

Se propone realizar una aplicación m\'ovil donde el usuario pueda hacer una reservaci\'on de una mesa para as\'i poder agilizar el proceso de reservar y que sea de una manera m\'as f\'acil y r\'apida para el usuario. 


\section{Objetivo general}

Desarrollar una aplicaci\'on m\'ovil que permita a los clientes reservar mesas del restaurante y realizar pedidos de comida a domicilio para contribuir al crecimiento del negocio y brindar a sus clientes una forma de adquirir sus platillos sin la necesidad de trasladarse al restaurante.


\section{Objetivos especificos}


\begin{itemize}
    \item Crear una web service para el desarrollo de la aplicaci\'on.
    \item Conectar la aplicaci\'on a una base de datos para el manejo de información
    \item Desarrollar una interefaz donde el usuario pueda ver promociones sin necesidad de registrarse para que el usuario tenga una visi\'on de lo que es el restaurante.  
   
    \end{itemize}