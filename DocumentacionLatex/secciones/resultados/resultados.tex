\chapter{Resultados y conclusiones}


Podemos concluir que el desarrollo de aplicaciones con Android studio es una herramienta muy completa para el desarrollo de aplicaciones nativas, ya que contiene diferentes componentes y herramientas que nos ayudan a un mejor diseño y funcionalidad de la aplicación. Sin embargo  tiene un nivel complicado para la diseño y desarrollo de aplicaciones ya que puede llegar a ser complicada la implementación de lo que si quiera desarrollar.

Para el desarrollo de esta aplicación hicimos uso de los diferentes componentes que proporciona Android Studio para el diseño y programación de la aplicación como el uso de fragments, activities, y elementos visuales, conectándolo con una base de datos desde firebase que lleva el control de los datos que dicha aplicación utiliza.
Como conclusión realizar una aplicación móvil en un restaurante es una gran idea tanto para el restaurante como para los clientes porque así los clientes que gusten reservar una mesa o hacer pedidos se les facilite por medio de su dispositivo móvil y que así los clientes puedan reservar las mesas que están disponibles e informarse sobre los platillos que sirve el establecimiento.