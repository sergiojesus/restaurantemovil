\chapter{Resultados y conclusiones}


Podemos concluir que el desarrollo de aplicaciones con Android studio es una herramienta muy completa para el desarrollo de aplicaciones nativas, ya que contiene diferentes componentes y herramientas que nos ayudan a un mejor diseño y funcionalidad de la aplicación. Sin embargo  tiene un nivel complicado para la diseño y desarrollo de aplicaciones ya que puede llegar a ser complicada la implementación de lo que si quiera desarrollar.

Para el desarrollo de esta aplicación hicimos uso de los diferentes componentes que proporciona Android Studio para el diseño y programación de la aplicación como el uso de fragments, activities, y elementos visuales, conectándolo con una base de datos desde firebase que lleva el control de los datos que dicha aplicación utiliza.
